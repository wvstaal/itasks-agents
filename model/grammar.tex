\section{Grammar}

\begin{grammar}
<agent> ::= `Agent' <id> <beliefs> <activities>

<beliefs> ::= `Beliefs' <belief>*

<belief> ::= <id> `:=' <expr>

<activities>  ::= `Activities' <activity>+

<activity> ::= <patterns> \{ `|' <expr> \} `=>' <actions>

<patterns> ::= <pattern> \{ `,' <pattern> \}*

<pattern> ::= <id> `<-' `task' `(' <id> `)' | <id> `<-' `action' `(' <id> `)'

<actions> ::= <action> \{ `,' <action> \}*

<action> ::= `act' `(' <id> `)'
\alt `edit' `(' <id> `,' <expr> `)'
\alt `updateBelief' `(' <var> `,' <expr> `)'

<id> ::= (`a'..`z')+

<expr> ::= (`0'..`9')+
\alt `"' (`a'..`z')* `"'
\alt <expr> = <expr> | <expr> + <expr> | <expr> - <expr> | <expr> * <expr>
\alt <expr> . <expr> | <expr> `[' <expr> `]' | `[' \{ <expr> \{ `,' <expr> \}* \} `]'
\alt <id>
\alt \lit{$\lnot$} <expr> | \lit{$\exists$} <id> \lit{$\in$} <expr>. <expr> | \lit{$\forall$} <id> \lit{$\in$} <expr>. <expr>
\alt `value' `(' <id> `)'
\alt `hasValue' `(' <id> `)' 
\end{grammar}

\section{Examples}

This example shows agents suitable for the 'planning dates' example in Bas Lijnse's thesis. The goal of participants is to select a date for a meeting. The coordinator proposes a few initial dates and other participants select one date. The coordinator makes the final decision.

\begin{alltt}
Agent Coordinator
Beliefs
    favoriteDate := "10-04-2013"
Activities
    enterDateTimes <- task(enterDateTimes), done <- action(enterDateTimesDone)
    => edit(enterDateTimes, ["10-04-2013", "11-04-2013"]), act(done)
	
    monitorDateTimes <- task(monitorDateTimes), decide <- action(makeDecision)
    | \(\exists v \in\) value(monitorDateTimes). v = favoriteDate
    => act(decide)
    
Agent DateChooser1
Activities
    chooseDate <- task(chooseDate), done <- chooseDateDone
    => edit(chooseDate, value(chooseDate)[0]), act(done)
    
Agent DateChooser2
Activities
    chooseDate <- task(chooseDate), done <- chooseDateDone
    => edit(chooseDate, value(chooseDate)[1]), act(done)
\end{alltt}

The agent \textit{Coordinator} is responsible for entering a few initial dates. When some other participant selects the coordinators favourite date, the agent makes the decision to pick that date. Agents \textit{DateChooser1} and \textit{DateChooser2} pick the first and second proposed date respectively. 